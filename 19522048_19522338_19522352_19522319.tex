\documentclass[a4paper,22pt]{article}
\usepackage[utf8]{vietnam}
\usepackage{enumitem}
\usepackage[T5]{fontenc}
\usepackage{amsmath}
\usepackage{amssymb}
\usepackage{makecell}
\usepackage{hyperref}
\usepackage{multirow}
\centerline{\LARGE HỢP ĐỒNG NHÓM}
\section*{\bfseries Các nội dung chính}
Tài liệu này là kết quả thảo luận và nhất trí của nhóm trong lần họp mặt đầu tiên, xác định các giá trị cốt lõi của nhóm:
\begin{itemize}
     \item Các nguyên tắc làm việc nhóm.
     \item Kế hoạch giao tiếp của nhóm.
     \item Các quy tắc thưởng và phạt của nhóm.
     \item Các tiêu chí đánh giá thành viên cuối môn học.
\end{itemize}
\section{Thông tin nhóm}
Tên nhóm: {\Large ACHIVEMENT}\\
Số lượng thành viên: 4 thành viên.\\
Mục tiêu: Làm đồ án cuối kì môn Kỹ năng nghề nghiệp.\\
\begin{tabular}{|c|c|c|c|}
\hline
STT & MSSV & Họ và tên & Email\\
\hline
1 & 19522352 & Võ Hoàng Nguyên Tín & 19522352@gm.uit.edu.vn\\
\hline
2 & 19522338 & Nguyễn Quang Tiến & 19522338@gm.uit.edu.vn\\
\hline
3 & 19522048 & Vũ Đình Bảo Phúc & 19522048@gm.uit.edu.vn\\
\hline
4 & 19522319 & Phạm Minh Thuận & 19522319@gm.uit.edu.vn\\
\hline
\end{tabular} \\
\textit {\LARGE- Phân công công việc:}
\begin{itemize}
     \item Nguyễn Quang Tiến: Nhóm Trưởng
     \item Phạm Minh Thuận: Tester
     \item Vũ Đình Bảo Phúc: Submit code 
     \item Võ Hoàng Nguyên Tín: R&D
\end{itemize}
\section{Các nguyên tác hoạt động nhóm}
\textit {Những điều thành viên trong nhóm phải làm theo}
\begin{itemize}
     \item Đi học, đi họp đúng giờ.
     \item Nghiêm túc trong lúc hoạt động nhóm.
     \item Có tinh thần trách nhiệm với nhóm, với tập thể.
     \item Biết lắng nghe và cho ý kiến.
     \item Cố gắng hoàn thành tốt công việc được giao.
     \item Luôn lấy khẩu hiệu "một người vì mọi người."
\end{itemize}
\newpage
\textit {Những điều thành viên trong nhóm không được làm}
\begin{itemize}
     \item Nghỉ học, nghỉ học không có lý do, không báo trước.
     \item Chửi tục, nói năng mất tôn trọng người khác.
     \item Gây mất đoàn kết trong nội bộ nhóm.
     \item Phải có chính kiến riêng của mình, không được ba phải.
     \item Công tư phân minh, không được để thù hằn cá nhân ảnh hưởng đến công việc và lợi ích nhóm.
\end{itemize}
\textit {Những điều thành viên trong nhóm nên làm theo}
\begin{itemize}
     \item Tham gia các hoạt động của nhóm lúc trên lớp và khi hoạt động nhóm.
     \item Nên có ý kiến riêng, suy nghĩ riêng của bản thân trong lúc hoạt động nhóm.
     \item Có thái độ tôn trọng, yêu thương và đoàn kết các thành viên trong nhóm.
     \item Nên có tinh thần học hỏi và lắng nghe ý kiến của người khác.
     \item Giúp đỡ các thành viên khác trong nhóm nếu có thể.
\end{itemize}
{\Large LƯU Ý:} mọi bất đồng và vấn đề phát sinh nên được thông báo cho trưởng nhóm để thảo luận và làm theo ý kiến của số đông.
\section{Kế hoạch họp nhóm}
Mỗi tuần trưởng nhóm sẽ thông báo thời gian họp nhóm cụ thể.\\
-Tần suất gặp nhau: Mỗi tuần 1 lần.\\ 
-Địa điểm: Trường đại học Công nghệ thông tin.\\
-Hình thức thông báo: qua tin nhắn facebook hoặc nhắn tin trực tiếp SMS.\\
-Hạn thông báo: trước 1 ngày.\\
-Thành viên khi nhận được tin nhắn thông báo phải hồi đáp lại để chứng tỏ đã nhận và đã đọc thông báo.\\
-Nếu thành viên không hồi đáp thông báo họp hoặc một thông báo bất kì thì nhóm trưởng sẽ gọi cho thành viên đó.
\section{Quy tắc thưởng và phạt}
\textit Các quy tắc thưởng:
\begin{itemize}
     \item Nếu hoàn thành tốt công việc nhóm giao, giúp thành viên khác hoàn thành công việc hoặc góp ý đưa ra những ý kiến tốt, có ý tưởng xuất sắc góp phần vào thành công của nhóm sẽ được ghi nhận và khen thưởng.
\end{itemize}
\textit {Các quy tắc phạt}
\begin{itemize}
     \item Nếu trễ họp 15 phút sẽ bị khiển trách và đóng phạt 10.000 vào quỹ nhóm (lần đầu), lần 2 sẽ bị loại ra khỏi buổi họp và đánh vắng buổi đó.
     \item Nếu được giao công việc nhưng không chịu làm(không nêu được lý do chính đáng) thì sẽ bị loại khỏi nhóm.
     \item Nếu giao công việc hoàn thành không đúng thời hạn sẽ bị trừ điểm vào tổng điểm đồ án.
\end{itemize}
\section{Tiêu chí đánh giá thành viên cuối môn học}
Sau khi hoàn thành mục tiêu của nhóm, các thành viên tự đánh giá bản thân về quá trình làm việc. Việc đánh giá sẽ ảnh hưởng đến điểm số của mỗi thành viên.\\
\hline
\centerline{\LARGE BẢNG NHẬN XÉT ĐÁNH GIÁ}
\textit {Dưới đây là những tiêu chí đánh giá, mỗi thành viên trong nhóm căn cứ vào các tiêu chí
sau và đánh giá xem mình đạt mức độ nào gồm 3 mức độ: Tốt/ Khá/ Chưa tốt. Sau đó
nhóm trưởng sẽ căn cứ vào các bài đánh giá và quá trình làm việc thực tế để đánh giá mọi thành viên, riêng về đánh giá
nhóm trưởng thì các thành viên còn lại sẽ đánh giá. Yêu cầu các bạn đánh giá một cách trung thực và khách quan. Cảm ơn các bạn!\\}
\begin{tabular}{|c|c|c|c|c|c|}
\hline
\makecell{Các tiêu\\ chí} & \makecell{Đi họp\\ nhóm đầy\\ đủ, đúng\\ giờ} & \makecell{Thái độ\\ nghiêm túc\\ trong làm\\ việc nhóm} & \makecell{Hoàn thành\\ nhiệm vụ\\ được giao\\ đúng hạn} & \makecell{Có tính\\ sáng tạo,\\ nội dung\\ hay} & \makecell{Nhóm \\ trưởng/\\thành viên\\đánh giá}\\
\hline
\makecell{Nguyễn Quang Tiến\\ 19522338} & x & x & x & x & Tốt \\
\hline
\makecell{Võ Hoàng Nguyên Tín\\19522352} & x & x & x & x & Tốt \\
\hline
\makecell{Vũ Đình Bảo Phúc\\ 19522048} & x & x & x & x & Tốt \\
\hline
\makecell{Phạm Minh Thuận\\ 19522319} & x & x & x &  & Khá \\
\hline
\end{tabular}
\newpage 

  \Large  Giới thiệu về chương trình/Game \\
        \begin{itemize}
            \large
            \item Kiểm tra chạm dưới, chạm trên, chạm trái, chạm phải, kiểm tra kết thúc trò chơi.
            \item Kiểm tra điểm đầu tiên của rắn xuất hiện, nếu trùng với mồi thì đổi chỗ.
            \item Kiểm tra ăn mồi. 
            \item Xoá đuôi ở những chỗ đã đi qua để di chuyển rắn.
            \item Vẽ mồi, vẽ rắn.
            \item Xác định toạ độ rắn xuất hiện.
            \item Hàm thêm độ dài cho rắn.\\\\
        \end{itemize}
    \Large Các điểm mà nhóm SV tâm đắc khi áp dụng các kỹ năng được biết trong việc xây dựng trò chơi này\\
        \begin{itemize}
           \large 
           \item \\Kỹ năng làm việc nhóm.
           \item Kỹ năng phân tích và giải quyết vấn đề.
           \item Kỹ năng tìm kiếm và thu thập thông tin thông qua Internet.\\\\

        \end{itemize}
    \Large Đánh giá việc thực hiện hợp đồng nhóm\\
        \begin{itemize}
            \large
            \item Các thành viên tuân thủ khá nghiệm ngặt những quy tắc mà nhóm đã đặt ra, tuy vẫn còn một vài trường hợp đến trễ buổi làm việc nhóm nhưng không đáng kể và các bạn đều có lý do chính đáng nên nhìn chung tất cả thành viên đều hoàn thành tốt trách nhiệm của mình với nhóm.\\
        \end{itemize}

\textit{Link nơi làm việc nhóm trên Github:https://github.com/uit19522338/ky-nang-nghe-nghiep }
\end{document}